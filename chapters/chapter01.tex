\chapter{Introduction}

The dawn of the digital age has brought fundamental changes to numerous areas of science as well as our everyday lives and in general, one may observe that many phenomena and aspects of the physical world - hardware - have been replaced by virtual simulations - software. This is also true for the realm of electronic music production. Whereas music-synthesizers such as the legendary “Moog” previously produced sounds through electronic circuitry, digital synthesizers can nowadays be implemented with common programming languages such as C, Java or C++. I aim to implement such a digital synthesizer with the C++ programming language.
