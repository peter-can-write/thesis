\chapter*{Introduction}
\addcontentsline{toc}{chapter}{Introduction}

The dawn of the digital age has brought fundamental changes to numerous areas of science as well as our everyday lives. In general, one may observe that many phenomena and aspects of the physical world --- hardware --- have been replaced by virtual simulations --- software. This is also true for the realm of electronic music production. Whereas music synthesizers such as the legendary "Moog" previously produced sounds through analog circuitry, digital synthesizers can nowadays be implemented with common programming languages such as C++. \vspace{\baselineskip}

The question that remains, of course, is: how? Where does one start on day zero? What knowledge is required? What resources are available off and online? What are the best practices? How can somebody with no previous experience in generating sound through software build an entire synthesis system, from scratch? Even though building a software synthesizer is itself a big enough mountain to climb, the real difficulty lies in finding the path that leads one up this mountain. \vspace{\baselineskip}

Two publications that are especially valuable when attempting to build a software synthesizer are \emph{BasicSynth: Creating a Music Synthesizer in Software}, by Daniel R. Mitchell, and \emph{The Scientist and Engineer's Guide to Digital Signal Processing}, by Steven W. Smith. The first book is a practical, hands-on guide to understanding and implementing the concepts behind digital synthesizers. It provides simple, straight-forward explanations as well as pseudo-code examples on a variety of subjects that are also discussed in this thesis. The real value of Mitchell's publication is that it simplifies and condenses years of industry practices and digital signal processing (DSP) knowledge into a single book. However, \emph{BasicSynth} abstracts certain topics, such as digital filters, too much. Where its explanations do not suffice for a full, or at least practical, understanding, \emph{The Scientist and Engineer's Guide to Digital Signal Processing} is an excellent alternative. Steven W. Smith's book explains concepts of Digital Signal Processing (DSP) in great detail, and is to some extent on the opposite end of the spectrum of explanation depth, when compared to \emph{BasicSynth}. It should be noted, however, that Smith's publication is not at all focused on audio processing or digital music, but on DSP in general. (Mitchell, 2008) (Smith, 1999) \vspace{\baselineskip}

This thesis is intended to dicuss the most important steps on the path from zero lines of code to a full synthesis system, implemented in C++. While providing many programming snippets and even full class definitions\footnotemark{}, a special focus will be put on explaining the fundamental ideas behind the programming, which very often lie in the realm of mathematics or general digital signal processing. The reason for this is that once a theoretical understanding has been established, the practical implementation becomes a trivial task.\vspace{\baselineskip}

\footnotetext{C++ program samples of up to 50 lines of code are displayed in-line with the text, longer samples are found in Appendix A.}

\noindent The first chapter will introduce some rudimentary concepts and phenomena of sound in the digital realm, as well as explain how digital sound differs to its analog counterpart. Subsequent chapters examine, among other things, how computer music is generated, modulated, filtered, synthesized and finally made audible. Images of sound samples in the time and frequency domain, as well as diagrams to abstract certain concepts, will be provided. Moreover, programming relationships, such as inheritance diagrams, between parts of the synthesizer implemented for this thesis, called \emph{Anthem}, are also displayed when relevant. Lastly, excerpts of email or online forum exchanges will be given when they contributed to the necessary knowledge. \vspace{\baselineskip}

It should be made clear that this thesis is not a "tutorial" on how to program a complete synthesis system in C++. It is also not designed to be a reference for theoretical concepts of digital signal processing. Rather, practice and theory will be combined in the most pragmatic way possible.
